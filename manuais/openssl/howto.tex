% Manual de Instalação e Configuração Subversion
%*************************************************
\documentclass[a4paper,10pt]{article}
\usepackage[brazil]{babel}
\usepackage[utf8]{inputenc}
%codigo
%\usepackage[latin1]{inputenc}
\usepackage{amsthm,amsfonts,amsmath,amssymb}
%img
\usepackage{graphicx}
\usepackage{subfig}
%tabela

%fim
%fim
\usepackage{makeidx}
\usepackage{listings}
\usepackage{enumerate}
\usepackage{hyperref}
\hypersetup{
  colorlinks,
  linkcolor=blue,
  filecolor=blue,
  urlcolor=blue,
  citecolor=blue
}
\lstset{
basicstyle=\small\ttfamily,
numbers=left,
numbersep=5pt,
xleftmargin=10pt,
frame=tb,
framexleftmargin=10pt
}

\renewcommand*\thelstnumber{\arabic{lstnumber}:}

\DeclareCaptionFormat{mylst}{\hrule#1#2#3}
\captionsetup[lstlisting]{format=mylst,labelfont=bf,singlelinecheck=off,labelsep=space}
%titulo
\title{Manual de instalação e configuração OpenSSL [Windows]}
\author{Leandro Kümmel Tria Mendes}
\makeindex
%inicio
\begin{document}
\maketitle
\begin{figure}[!htb]
  \centering
  \includegraphics[scale=0.5]{../img/logo.png}
\end{figure}
\newpage
\tableofcontents
\listoffigures
\listoftables
\section{Pré-requisitos}
Para a instalação e configuração do sistema que gera certificados digitais é necessária a instalação do OpenSSL\footnote{\url{http://slproweb.com/products/Win32OpenSSL.html}}. A máquina alvo deverá conter a última versão do Apache (versão $\ge$ 2.2) \textbf{com módulo de ssl}\footnote{\url{http://ftp.unicamp.br/pub/apache//httpd/binaries/win32/httpd-2.2.25-win32-x86-openssl-0.9.8y.msi}} 
\section{Instalação e Gerar Certificado}
\begin{enumerate}[I]
\item \label{itm:install} \textbf{Instalação}: A instalação do OpenSSL é básica, simples next-next-finish (NNF).\textbf{A instalação deverá, por padrão, ser realizada no diretório raiz C:/}. Após a instalação teremos a seguinte pasta: \textbf{C:/Openssl/}
\item \label{itm:createcert} \textbf{Gerar certificado}: Para gerar um novo certificado devemos abrir o DOS (iniciar$->$executar: digite $"cmd"$ e enter) navegar até a pasta de instalação[\ref{itm:install}] entre em \textbf{/bin} e digitar os comando abaixo.
 \begin{lstlisting}[caption={Gerar chave e certificado}]
   # gerar arquivo com a chave privada.
   openssl genrsa -aes256 -out privada.key 2048 
   # gerar a chave publica.
   openssl rsa -in privada.key -out publica.key 
   # gerar o certificado digital utilizando a configuracao padrao do openssl.
   openssl req -new -x509 -nodes -sha1 -key publica.key \
   -out publica.crt -days 999 -config C:\Openssl\share\openssl.cnf
 \end{lstlisting}
 A chave pública e o certificado encontram-se em \textbf{C:/Openssl/bin/}  publica.key e publica.crt, respectivamente.
\end{enumerate}
\section{Configurar Apache + SSL}
Nessa seção será necessário a cópia de alguns arquivos para a pasta \textbf{system32} do windows e a configuração de alguns arquivos do apache\footnote{Lembre-se que a instalação do apache deverá ser feita com uma versão acima da 2.2 e com OpenSSL nativo.}
\section{System 32}
Navegue nas pastas até a instalação do Apache e procure na pasta \textbf{bin} as .dll \textbf{ssleay32.dll} e \textbf{libeay32.dll}. Copie as mesmas para a pasta \textbf{C:/Windows/system32}\footnote{Esse caminho pode variar, dependendo da versão do S.O. e de sua instalação}
\section{Apache}
Para esse item é necessário lembrar a pasta de instalação do Apache, assim como sua pasta de configuração. A instalação, exemplo, está localizada em \textbf{S:/Apache/}.\footnote{Vide: engenheiro.cpo.unicamp.br}.\\
A cada conexão no Apache haverá a troca de certificados e chave pública entre o cliente e servidor. Para isso deveremos criar uma pasta localizada em \textbf{S:/Apache/conf/extra/}, nominada \textbf{ssl}. Resultando em um nova pasta \textbf{S:/Apache/conf/extra/ssl/}.\\
Lembrando o item \ref{itm:install}, temos a chave e certificado já gerados e presentes em \textbf{C:/Openssl/bin/}. Copie os arquivos criados (publica.key e publica.crt) para a nova pasta criada acima. Concluindo, ao navegar na pasta \textbf{/conf/extra/ssl/} do Apache, deveremos encontrar dois arquivos [\ref{itm:createcert}], um com extensão \textbf{.key} e o outro com \textbf{.crt}.
\begin{itemize}
\item \textbf{httpd-ssl.conf}: Arquivo localizado em \textbf{S:/Apache/conf/extra/}
 \begin{lstlisting}[caption={httpd-ssl.conf}]
   # Localize SSLSessionCache e troque seu conteudo por
   SSLSessionCache "shmcb:S:/Apache/logs/ssl_scache(512000)" 

   # Localize SSLCertificateFile e coloque o caminho do certificado 
   SSLCertificateFile "S:/Apache/conf/extra/ssl/publica.crt" 

   # Localize SSLCertificateKeyFile e coloque o caminho da chave publica 
   SSLCertificateKeyFil "S:/Apache/conf/extra/ssl/publica.key" 

   # Verifique se o SSLMutex esta como default
   SSLMutex default

   # Localize <VirtualHost> e troque seu conteudo por
   DocumentRoot "S:/htdocs" # diretorio dos projetos
   ServerName engenheiro.cpo.unicamp.br:443 # servidor
   ServerAdmin admin@localhost # email do administrador
   ErrorLog "S:/Apache/logs/ssl_error.log" # log de erro
   TransferLog "S:/Apache/logs/ssl_access.log" # log de transferencia

   # Localize a linha SSLOptions +StdEnvVars
   # Substitua por
   SSLOptions +StdEnvVars
   Options Indexes FollowSymLinks MultiViews
   AllowOverride All
   Order allow,deny
   allow from all

   # Localze a linha CustomLog e troque seu conteudo por
   CustomLog "S:/Apache/logs/ssl_request.log" \ 
   "%t %h %{SSL_PROTOCOL}x %{SSL_CIPHER}x \"%r\" %b" 
 \end{lstlisting}
\item \textbf{httpd.conf}:  Arquivo localizado em \textbf{S:/Apache/conf/}
  \begin{lstlisting}
    Procure por mod_ssl.so e descomente a linha, caso esteja comentada.
    Ache a linha com extra/httpd-ssl.conf e tambem a descomente.
  \end{lstlisting}
Reinicie o serviço do apache e pronto! Para testar acesse o DocumentRoot através de seu browser utilizando https:// ao invés de http://.
\end{itemize}
\end{document}
