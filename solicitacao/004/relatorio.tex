%*************************************************
\documentclass[a4paper,10pt]{article}
\usepackage[brazil]{babel}
\usepackage[utf8]{inputenc}
%codigo
%\usepackage[latin1]{inputenc}
\usepackage{amsthm,amsfonts,amsmath,amssymb}
%img
\usepackage{graphicx}
\usepackage{subfig}
%tabela

%fim
%fim
\usepackage{listings}
\usepackage{makeidx}
\usepackage{enumerate}
\usepackage{hyperref}
\hypersetup{
  colorlinks,
  linkcolor=blue,
  filecolor=blue,
  urlcolor=blue,
  citecolor=blue
}

%titulo
\title{Solicitação 004}
\author{Leandro Kümmel Tria Mendes}
\makeindex
%inicio
\begin{document}
\maketitle
\begin{figure}[!htb]
  \centering
  \includegraphics[scale=0.5]{logo.png}
\end{figure}
\newpage
\section{Introdução}
Será implementado um alerta, apenas para um grupo de usuários, para os contratos quando estão próximos da data, aditivada, de conclusão prevista.
\section{Requisitos}
\subsection{Aviso}
Aparecerá na página principal do SigPod \url{http://sg.cpo.unicamp.br/} um alerta, como se fosse um pop-up, chamado de \emph{jquery.ui-dialog}.
\begin{itemize}
\item \emph{Ativar}: Para ativar o \textbf{aviso}, de que falta \boxed{X} tempo (dias) para a validade do contrato. O controle da variável \boxed{X} será feito pelo painel de \emph{Administração}
\url{http://sg.cpo.unicamp.br/adm.php}, o valor dessa valerá para todos os contratos.
\item \emph{Desativar}: Quando um usuário pertencente ao grupo \boxed{U}, o qual o controle também é feito no painel de \emph{Administração}, desejar desabilitar o \textbf{aviso} a ação será
propagada para todos os usuário pertencentes a \boxed{U}. Para desabilitar bastará clicar no botão \boxed{[Desabilitar]}.
\end{itemize}
\subsection{Extras}
\begin{itemize}
\item \emph{Destacar data}: Na tela em que os resultados da busca por um ou mais contratos são exibitos, destacar a data de conclusão prevista, para aqueles que estão próximos do vencimento.
\item \emph{Habilitar busca por vencimento}:  Disponibilizar na tela de busca, para contratos, um botão que habilita a busca apenas por contratos próximos da data de conclusão ou já vencidos.
\end{itemize}
\end{document} 
