%*************************************************
\documentclass[a4paper,10pt]{article}
\usepackage[brazil]{babel}
\usepackage[utf8]{inputenc}
%codigo
%\usepackage[latin1]{inputenc}
\usepackage{amsthm,amsfonts,amsmath,amssymb}
%img
\usepackage{graphicx}
\usepackage{subfig}
%tabela

%fim
%fim
\usepackage{listings}
\usepackage{makeidx}
\usepackage{enumerate}
\usepackage{hyperref}
\hypersetup{
  colorlinks,
  linkcolor=blue,
  filecolor=blue,
  urlcolor=blue,
  citecolor=blue
}

%titulo
\title{Solicitação 003}
\author{Leandro Kümmel Tria Mendes}
\makeindex
%inicio
\begin{document}
\maketitle
\begin{figure}[!htb]
  \centering
  \includegraphics[scale=0.5]{logo.png}
\end{figure}
\newpage
\section{Introdução}
Implementar mudanças no contratos, são elas:\\
\textbf{Mudanças válidas apenas para Valor do Projeto, Valor de Mão-de-obra, Valor do Material, Prazo Contratual, Prazo de Conclusão Previsto Projeto/Obra}
\begin{itemize}
\item \label{itm:button_aditivar} \emph{Botão [Aditivo]}: Trocamos de lugar, agora aparece na coluna à esquerda abaixo do nome da cada campo (Valor do Projeto etc...). Também foi criado um novo campo \boxed{<input type="text" id="aditivar\_valor\_moeda" mask="true">} o qual recebe uma
máscara, para valores monetários (jquery) arquivo scripts/plugins/mascara.jquery.js. Para valores inteiros, ou seja, para o total de dias nada modificamos, apenas há um controle de qual input devemos deixar aparecendo, se o com máscara monetária, ou o simples que recebe os dias.\\
Por último, foi solicitado que o campo \textbf{motivo} ficasse estático, com valor \textbf{outro}, podendo ser alterado.
\item \label{itm:button_editar} \emph{Botão [Editar]}: O botão não aparecia nos aditivos monetários, o mesmo foi habilitado para os usuários.
\item \label{itm:interface} \emph{Display dos dados}: No antigo display, os aditivos apareciam na seguinte configuração \boxed{30 (25,00 \%) (Motivo: atraso na obra) [Editar]}, um abaixo do outro. Foi alterado com a seguinte configuração \boxed{(+) 53 dias(10,60\%) [Editar]}, para ver mais sobre o mesmo basta um click em cima do aditivo.
\item \label{itm:somas} \emph{Valor total e Data de conclusão}: Antes os valores não eram alterados, agora os mesmos são incrementados pelos aditivos. A Data de conclusão com os incrementos também foi atualizada na busca principal pelos contratos. Há um total de cada aditivo ao final das descrições de cada campo, que contém aditivo.
\end{itemize}
\textbf{Mudanças válidas apenas para Início Projeto/Obra}
\begin{itemize}
\item \label{itm:datepicker} \emph{Campo Início Projeto/Obra}: Foi implementado um datepicker do jquery.ui para esse campo.
\end{itemize}
\section{Arquivos alterados}
Busque por \"Solicitacao 003\" ou apenas \"003\", para ver os comentários e os métodos/funções alterados.
\begin{itemize}
\item \emph{classes/BD.php}
\item \emph{classes/Contrato.php} : (maioria das modificações já que controla o frontend(html))
\item \emph{sgd.php}
\item \emph{sgd\_busca.php}
\item \emph{sgd\_module.php}
\item \emph{includeAll.php}
\item \emph{scripts/busca\_doc2.js}
\item \emph{scripts/sgd\_contrato.js} 
\item \emph{css/geral.css} 
\end{itemize}
\section{Arquivos novos}
Sempre procure comentários nas interfaces primeiro, lembre-se que elas são um descritor dos métodos públicos da classe a qual implementa a(s) mesma(s).
\begin{itemize}
\item \emph{BD/DAO/interfaces/DAO.class.php} : DAO significa DataAcessObject, ou seja, acesso ao banco de dados, com ele queremos reunir métodos que agilizem e otimizem as querys.
\item \emph{classes/contrato/Aditivo.class.php} 
\item \emph{classes/contrato/interfaces/AditivoDAO\_IF.class.php} 
\item \emph{classes/contrato/interfaces/AditivoIF.class.php} 
\item \emph{classes/contrato/dao/AditivoDAO.class.php} 
\item \emph{classes/frontend/html/} : Esse pacote foi criado para auxiliar na geração do código html. Importante notar que os métodos setNext() e setChildren() utilizam métodos de fila, ou seja, primeiro a entrar é o primeiro elemento da fila.
\item \emph{classes/frontend/html/HtmlTable.class.php} 
\item \emph{classes/frontend/html/HtmlTag.class.php} 
\item \emph{classes/frontend/html/HtmlTagStyle.class.php} 
\item \emph{classes/frontend/html/HtmlTagAttr.class.php} 
\item \emph{classes/frontend/html/interfaces/HtmlTagIF.class.php} 
\item \emph{classes/frontend/html/interfaces/HtmlTagStyleIF.class.php} 
\item \emph{classes/frontend/html/interfaces/HtmlTagAttrIF.class.php} 
\item \emph{script/plugins/mascaras.jquery.js} : Contém Masked Input plugin for jQuery e MaskMoney, mais o método \$.fn.mascara(tipo)
\end{itemize}
\end{document} 
