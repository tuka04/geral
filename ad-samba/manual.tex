%*************************************************
\documentclass[a4paper,10pt]{article}
\usepackage[brazil]{babel}
\usepackage[utf8]{inputenc}
%codigo
%\usepackage[latin1]{inputenc}
\usepackage{amsthm,amsfonts,amsmath,amssymb}
%img
\usepackage{graphicx}
\usepackage{subfig}
%tabela

%fim
%fim
\usepackage{listings} 
\usepackage{makeidx}
\usepackage{enumerate}
\usepackage{hyperref}
\hypersetup{
  colorlinks,
  linkcolor=blue,
  filecolor=blue,
  urlcolor=blue,
  citecolor=blue 
}

%titulo
\title{Configuração do domínio (Active Directory) no linux}
\author{Leandro Kümmel Tria Mendes}
\makeindex
%inicio
\begin{document}
\maketitle
\begin{figure}[!htb]
  \centering
  \includegraphics[scale=0.5]{logo.png}
\end{figure}
\newpage
\section{Introdução}
Vamos configurar o domínio em uma máquina linux rodando a distribuição FedoraCore 15+, ou seja, válido para versões maiores que a 15.

Todos os comandos referem-se a distribuições da RedHat, ou seja, para Debian, Ubuntu pequenas mudanças, tal como, localização dos arquivos de configuração
ocorrerão, para isso consulte \url{http://www.google.com.br}
\section{Instalações}
Para colocar a máquina linux no domínio precisamos instalar alguns pacotes:
\begin{itemize}
\item \label{itm:kerberos} \emph{Kerberos}: Kerberos é o nome de um Protocolo de rede, que permite comunicações individuais seguras e identificadas, em uma rede insegura. 
O Kerberos previne Eavesdropping e Replay attack, e ainda garante a integridade dos dados. \footnote{\url{http://pt.wikipedia.org/wiki/Kerberos}}
\item \label{itm:openldap} \emph{OpenLDAP}: O OpenLDAP é um software livre de código aberto que implementa o protocolo LDAP. Ele é um serviço de diretório baseado no padrão X.500.
O OpenLDAP é independente de sistema operativo. Várias distribuições Linux incluem o pacote do OpenLDAP. O software também corre nos sistemas operativos BSD, AIX, HP-UX, Mac OS X, Solaris, Microsoft Windows (2000, XP, 2003, 2008, Vista, win7 e win 8) e z/OS. \footnote{\url{http://pt.wikipedia.org/wiki/OpenLDAP}}
\item \label{itm:cups} \emph{CUPS}: Um computador rodando o CUPS é um hospedeiro que pode aceitar tarefas de impressão de computadores clientes, processá-los e enviá-los à impressora correta, além disso é possível monitorar impressões, relatar erros de impressões, visualizar relatórios sobre número de páginas impressas, data e horário da mesma.\footnote{\url{http://pt.wikipedia.org/wiki/CUPS}}
\item \label{itm:samba} \emph{SAMBA}: Samba é  utilizado em sistemas operacionais do tipo Unix, que simula um servidor Windows, permitindo que seja feito gerenciamento e compartilhamento de arquivos em uma rede Microsoft. Na versão 3, o Samba não só provê arquivos e serviços de impressão para vários Clientes Windows, como pode também integrar-se com Windows Server Domain, tanto como Primary Domain Controller (PDC) ou como um Domain Member. Pode fazer parte também de um Active Directory Domain. \footnote{\url{http://pt.wikipedia.org/wiki/Samba_(servidor)}}
\end{itemize}

\subsection{Procedimento}
\subsubsection{Kerberos}
\$  \textbf{sudo yum install krb5-appl-clients.i686 krb5-appl-servers.i686 krb5-pkinit-openssl.i686 krb5-workstation.i686}
\subsubsection{OpenLDAP}
\$  \textbf{sudo yum install openldap.i686 openldap-clients.i686  pam\_ldap.i686 smbldap-tools.noarch}
\subsubsection{CUPS}
CUPS já vem instalado nativo no FedoraCore 15+
\subsubsection{SAMBA}
\$  \textbf{sudo yum install samba-winbind.i686 samba-client.i686 samba-common.i686 samba-winbind-clients.i686 samba-winbind-krb5-locator.i686}
\section{Configurações}
Para todas as configurações mostradas a baixo, utilizamos uma máquina com nome \emph{CPO75}, o domínio é \emph{CPO.UNICAMP.BR} de IP \emph{143.106.193.3}.
\subsection{Kerberos}
Editar o arquivo localizado em /etc/krb5.conf (adicionar ou alterar as informações existentes)\\
\$ \textbf{sudo nano /etc/krb5.conf}
\\
\lstinputlisting{krb5.conf}
Testar a configuração, primeiro teste com uma senha errada, e esta deve dar uma mensagem de erro, depois teste com a senha correta, e nada deve ocorrer.
\$ \textbf{kinit administrador}

\subsection{WinBind e NS}
Alterar as linhas do arquivo /etc/nsswitch.conf.\\
\$ \textbf{sudo nano /etc/nsswitch.conf}\\
\lstinputlisting{nsswitch.conf}
\subsection{Samba}
Alterar o arquivo localizado em /etc/samba/smb.cfg\\
\$ \textbf{sudo nano /etc/samba/smb.cfg}
\lstinputlisting{smb.conf}
Teste a configuração\\
\$ \textbf{testparm}
\subsection{AuthConfig}
Para registrar algumas configurações.
\$ \textbf{sudo authconfig --updateall --enablewinbind --enablewinbindauth --enablewinbindusedefaultdomain}
\subsection{Hosts e DNS}
Precismos configurar o hostname, e alias, da máquina e seu DNS (windows têm problemas de comunicação com linux)
\subsubsection{Hosts}
Altere o arquivo localizado em /etc/hosts, alterando o nome da máquina e seu alias.
Troque a linha que contém 127.0.0.1 pela apresentada logo abaixo.
\$  \textbf{sudo nano /etc/hosts}
\lstinputlisting{hosts.conf}
\subsubsection{resolv.conf - DNS}
Altere o arquivo localizado em /etc/resolv.conf , nele devem conter no minimo as três linhas
abaixo, na ordem apresentada.
\lstinputlisting{resolv.conf}
\section{Iniciar serviços}
Rode os três comandos a seguir.\\
\$ \textbf{smbd -D}\\
\$ \textbf{nmdb -D}\\
\$ \textbf{winbindd -D}\\
\section{Trabalhando em rede}
Se tudo ocorreu bem até agora, faltam poucos passos, porém os mais problemáticos.
\subsection{Net ADS join}
Digite o comando abaixo.\\
\$ \textbf{sudo net ads join -U administrator}\\
Caso encontre o erro: \emph{Failed to join domain: failed to lookup DC info for domain 'CPO.UNICAMP.BR' over rpc: Logon failure}, prossiga sem medo.
\subsection{smbclient}
Digite o comando abaixo, pode se utilizar qualquer usuário, não necessáriamente o administrador.\\
\$ \textbf{sudo smbclient -U administrador -L 143.106.193.3}\\
Coloque a senha e o resultado esperado é algo parecido com:
\lstinputlisting{result.smbclient}
É possível utilizar o alias arquiteto, para o endereço 143.106.193.3\\
\$ \textbf{sudo smbclient -U administrador -L arquiteto}\\
\section{Montar/Mapear a máquina arquiteto}
Primeiro precisamos criar um diretório, que chamaremos de area\_transf e monta-lo em /mnt/\\
\$ \textbf{sudo mkdir -p /mnt/area\_transf}\\
\$ \textbf{sudo mount -t cifs -o rw,noperm,user=<USUARIO>,password=<SENHA> //arquiteto/area\_transf /mnt/area\_transf}\\
Por último, para testar.\\
\$ \textbf{sudo ls -l /mnt/area\_transf}\\
\end{document} 
